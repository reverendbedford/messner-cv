\usepackage{ifthen}
\usepackage[sorting=ydnt,style=numeric,defernumbers]{biblatex}
\addbibresource{publications.bib}

\begin{document}

\cvname{Mark Christian Messner}

\institution{Argonne National Laboratory}

\contactaddress{Building 212\\
9700 Cass Ave.\\
Lemont, IL 60439}

\phonenumber{+1 (630) 252-6510}

\email{messner@anl.gov}

\personalemail{mark.messner@gmail.com}

\homephonenumber{+1 (847) 254-1740}
\makecvtitle

\cvsection{Education}

\textbf{Doctor of Philosophy}, \emph{University of Illinois at Urbana-Champaign},
GPA 3.96/4.00\years{2011-2014}\\
Major: Civil and Environmental Engineering\\
Advisor: Robert Dodds, Jr. \\
Dissertation: \textit{Micromechanical models of delamination in Al-Li
alloys}\\
Computational Science and Engineering Certificate

\textbf{Master of Science}, \emph{University of Illinois at Urbana-Champaign},
GPA 3.96/4.00 \years{2010-2011}\\
Major: Civil and Environmental Engineering\\
Advisor: Robert Dodds, Jr. \\
Computational Science and Engineering Certificate

\textbf{Bachelor of Science}, \emph{University of Illinois at Urbana-Champaign},
GPA 3.97/4.00 \years{2006-2010}\\
Major: Civil and Environmental Engineering, Minor: German\\
Degree awarded with Highest Honors and University Honors 

\cvsection{Appointments}

\textbf{Principal Mechanical Engineer}, \emph{Argonne National Laboratory}
\years{2016-}\\
Research topics: High temperature structural materials, design
of high temperature nuclear reactors and concentrating solar power systems,
crystal plasticity, machine learning
methods for materials and material constitutive modeling,
qualification of AM nuclear components

Supervised a teams of 2-3 postdocs and 1-2 staff on projects
supported by the U.S. Department of Energy, Office of Nuclear Energy and 
the Office of Energy Efficiency and Renewable Energy
and the U.S. Nuclear Regulatory Agency

Managed a research portfolio with greater than \$2 million per year in total funding.  

Initiated several work on high temperature material simulation and qualification through several different DOE:NE programs.

Initiated funded projects in new topic areas on concentrating solar power and advanced manufacturing

Led work on the revision and improvement of several parts of the American Society of 
Mechanical Engineers Boiler and Pressure Vessel Code, Section III, Division 5 covering the
design and construction of high temperature nuclear reactors

\textbf{Postdoctoral Researcher}, \emph{Lawrence Livermore National
Laboratory} \years{2014-2016}\\
Supervisor: Nathan Barton \\
Research topics: Multiscale material modeling of additively manufactured
structured materials, modeling and optimization of lattice-structured
meta-materials, multiscale modeling of HCP metals

\textbf{Research Assistant}, \emph{University of Illinois at Urbana-Champaign}
\years{2010-2014}\\
Supervisor: Robert Dodds, Jr. \\
Research topics: Parallel performance of WARP3D, crystal plasticity,
mesoscale modeling of fatigue/fracture processes, homogenization and
multiscale damage calculations

\cvsection{Honors/Awards}

Doug Scarth Early Career Leadership Award for Outstanding Service to the Pressure Vessels \& Piping (PVP) Division as an Early Career Engineer \years{2023}

Secretary of Energy Achievement Award \years{2022}

ASME Boiler \& Pressure Vessel Code Certificate of Acclamation \years{2022,2023}

Impact Argonne Award for Innovation \years{2020}

National Defense Science \& Engineering Graduate Fellowship \years{2012-14}

University Fellowship \years{2010-11}

Tau Beta Pi \years{2008-}

\cvsection{Professional Affiliations}

ASME \years{2016-}

ANS \years{2016-}

APS \years{2014-2016}

ASCE, EMI \years{2006-}

\cvsection{Professional Service}

Generation IV Forum: \years{2018-} Task Group on Advanced Manufacturing and Materials Engineering \emph{co-chair} 

ASME Boiler \& Pressure Vessel Code \emph{committee chair}: \years{2020-} BPV III WG on Analysis Methods

ASME Boiler \& Pressure Vessel Code \emph{committee chair}: \years{2018-2020} BPV III SWG on Inelastic Analysis Methods 

ASME Boiler \& Pressure Vessel Code \emph{committee member}: \years{2017-} BPV III SG High Temperature Reactors,
WG on Analysis Methods, WG High Temperature Flaw Evaluation, 
WG Creep-Fatigue and Negligible Creep; BPTCS/BNCS Special Committee on 
Use of Additive Manufacturing for Pressure Retaining Equipment, BVP I/VIII WG on Elevated Temperature Design

PVP Conference: \years{2017-} Technical Program Representative, co-Technical Program Representative, Honors and Awards Chair, Materials and Fabrication Secretary, track co-chair

WCCM/USNCCM: \years{2018-2019} \emph{track organizer} 

\emph{Reviewer for (past year)}: \years{2023}  
Materials Chemistry and Physics
International Journal of Solids and Structures
Computer Methods in Applied Mechanics and Engineering
Engineering Fracture Mechanics
International Journal of Fracture
Advanced Engineering Materials
International Journal of Fatigue
Journal of Dynamic Behavior of Materials
PVP Conference Proceedings

External proposal reviewer for DOE:NE, DOE:EERE, DOE:FES, and the NSF. \years{2018-}

\cvsection{Institutional and Community Service}

Library User Committee member \years{2019-}

Volunteer at middle school/high school DOE Science Bowl \years{2015-}

STEM chat volunteer for local elementary and high schools \years{2020-}

Undergraduate and graduate student summer research program mentor \years{2017-2018, 2021-2023}

Qualification exam review course, course organizer \years{2013-2014}

\cvsection{PhD Committee Service}

Alon Katz (Georgia Institute of Technology) \years{2018-2021} 

Janzen Choi (University of New South Wales) \years{2022-}

\cvsection{DOE:NE Work Packages Managed}

ART: Several work packages -- ~\$400k/year \years{2021-}

NEAMS: Structural Materials -- ~\$300k/year \years{2019-}

AMMT: Several work packages -- ~\$500k/year \years{2022-}

NDMQi: High Temperature Qualification -- ~\$200k/year \years{2020-2021}

\cvsection{Funding Awards as PI}

EPRI: Accelerating qualification of advanced material -- \$80k \years{2023-2024}

US NRC: Technical Assessment of ASME BPVC Section III, Div. 5 Composites Rules -- \$360k \years{2023-2024}

Argonne Laboratory Directed Research and Development (LDRD): Microarchitected Composites -- \$300k \years{2022-2024}

US NRC: Technical Assistance Pertaining to Advanced Reactors - Assessment of Salt Properties, Stress Relaxation Cracking and Materials/Component Integrity -- \$150k \years{2022-2024}

DOE:HPC4Energy: An ICME Modeling Framework for Metal Matrix Composites Focusing on Ultrahigh Temperature Matrix Material with Tungsten Carbide Reinforcement -- \$300k \years{2021-2022}

DOE:EERE: Design Methods, Tools, and Data for Ceramic Solar Receivers -- \$955k \years{2021-2023}

DOE:EERE: High Temperature Receiver Design Package -- \$517k \years{2020-2021}

US NRC: Assess State of Knowledge of Modeling and Simulation and Microstructural Analysis for Advanced Manufacturing Technologies (AMTs) -- \$200k \years{2019-2020}

DOE:NE FOA: Modeling and Simulation Development Pathways to Accelerate KP-FHR Licensing (topic PI) -- \$500k \years{2019-2021}

DOE:EERE Gen3 CSP: Creep-fatigue design for CSP receivers (topic PI) -- \$375k \years{2018-2020}

LLNL TechBase: Adaptive smart materials -- \$65k \years{2016}

LLNL TechBase: Material model library for lattice structured meta-materials
-- \$50k \years{2015}

\extra

\cvsection{Publications/Presentations}

\nocite{*}

\cvsubsection{Refereed journal publications}

\printbibliography[keyword=refereed,heading=none]

\cvsubsection{Pending refereed journal publications}

\printbibliography[keyword=pending,heading=none]

\cvsubsection{Refereed conference publications}

\printbibliography[keyword=conf,heading=none]

\cvsubsection{Patents}

\printbibliography[keyword=patents,heading=none]

\cvsubsection{Non-refereed publications}

\printbibliography[keyword=nonref,heading=none]

\cvsubsection{Invited talks}

\printbibliography[keyword=invited,heading=none]

\cvsubsection{Numerous conference presentations}

\end{document}
